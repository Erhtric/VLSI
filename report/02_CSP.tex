\section{Constraint Satisfaction Programming - CSP}

\subsection{Model Description}

In order to bear the problem formulation in MiniZinc, the model chosen simply describes the circuits by means of rectangles. In particular, we are given a set of \emph{N} rectangular pieces, ideally representing our circuits, each of a predefined width $w_i$ and height $h_i$, where $i \in 1, \dots, N$. Each rectangle is handily represented by a tuple $(x_i, y_i)$ with $x_i \in 0,\dots, \textit{WIDTH}$ and $y_i \in 0,\dots, \textit{HEIGHT}$, namely the coordinates of the left-bottom corner. The objective is to find the smallest bounding-box, the silicon plate, that contains these rectangles without any overlapping between them. In this specific project the width of the box is passed as a parameter, let it be \textit{WIDTH}, so the objective simplifies to a minimization of the height, call it \textit{HEIGHT}, of the bounding-box. In the first formulation we assume that the rectangles have fixed orientation, so any rotation is not allowed. 

\bigskip
Without going into the implementation details, we will define the constraints to be satisfied by means of logical propositions. First of all we have to explicit the constraints that define boundaries in which the rectangles must be placed. Given two circuits $c_i$, positioned in $(x_i, y_i)$, and $c_j$, positioned in $(x_j, y_j)$, each one with its height and width; we could formulate the \textit{boundary constraints} as follows:
\begin{align}
    &\forall{i \in  1, \dots, N}.( x_i + w_i \leq \textit{WIDTH}),\\
    &\forall{i \in  1, \dots, N}.( y_i + h_i \leq \textit{HEIGHT}).
\end{align}

These constraints ensure that the circuits stays in the plate.

\clearpage